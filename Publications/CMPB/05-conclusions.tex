\section{Discussion}
\label{sec:discussion}
We here present \gls*{mbis}, a segmentation tool particularly designed for
  multivariate data, and based on the Bayesian framework.
\Gls*{mbis} includes as main methodological novelties a new approach to 
  bias correction and the \gls*{markov} model optimization using
  \gls*{graph_cuts}.
After reviewing the theoretical background and implementation details,
  we reported an study evaluating the accuracy, with comparison to a widely
  used similar tool (\gls*{fast}).
Finally, we demonstrated the robustness of \gls*{mbis} on two publicly available
  multivariate databases.

\subsection{Accuracy performance}
Both visual (\autoref{fig:brainweb_hard}, \autoref{fig:brainweb_fuzzy}) and
  quantitative (\autoref{table:brainweb}) results showed the accuracy
  of the tool.
We recall that this claim was restricted to only one model from a synthetic brain
  database.
Many studies \citep{cuadra_comparison_2005,de_boer_accuracy_2010,ashburner_unified_2005,
  klauschen_evaluation_2009} that evaluated the accuracy of monospectral segmentation
  methods on \gls*{t1} \gls*{mri} have been reported.
In general, these studies used the 20 normal models from the BrainWeb 
  \citep{aubert-broche_twenty_2006}, evaluation tools \citep{shattuck_online_2009},
  or a number of manually segmented studies as ground-truth.
Nonetheless, the BrainWeb database only provides multivariate datasets for one
  single model, manual segmentation is unaffordable in multivariate data, and
  to our knowledge, there are no other evaluation resources of multivariate images.
Thus, there is an important lack of realistic ground-truth data to test
  multivariate segmentation of the brain.
Additionally, quantitative assessment of accuracy can be discredited in two ways.
First, synthetic models may not capture the unpredictable complexity of the real
  data supplied by a healthy or diseased human brain (i.e. foldings, \gls*{mri} 
  contrast properties).
Second, manual segmentation of real data taken as gold-standard
  is unaffordable, or at least, prone to inaccuracy and inconsistency
  that are intrinsic to the methodology itself and the effort-demanding nature of manual
  segmentation.
Therefore, evaluation experiments based on this scheme are illustrative
   but not definitive.
As an informal corollary, we claim that once the segmentation accuracy has been 
  assessed, it is equally or even more important to explore the challenging issue 
  of repeatability of results.
The reproducibility problem has become a main focus of interest 
  \citep{landman_effects_2007,de_boer_accuracy_2010} in every medical imaging study 
  because its absence has more negative consequences than inaccuracy itself.

Besides the segmentation results, we presented a new application of a B-spline model for
  the bias estimation.
Built upon two existing methods, we combined the original bias estimation methodology
  described previously \citep{van_leemput_automated_1999} with the bias field model proposed
  in another study \citep{tustison_n4itk:_2010}.
The B-spline model of \citeauthor{tustison_n4itk:_2010} has been proven to behave accurately
  without a heavy computational cost, and it was naturally embedded within the \gls*{e_m} 
  algorithm, as described in \autoref{sec:em}.
Visual assessment of results was documented.
Our methodology improved the bias estimated with respect to \gls*{fast} for all the channels.
Moreover, multivariate segmentation performed robustly against the bias field.
This is justified because the image channels share a unique distribution model that is used to
  estimate the bias more effectively, regardless of the modality of the channel.
Even though we observed that the bias correction can have a slightly negative impact
  on final segmentation, we concluded that the explicit modeling of the bias field is
  interesting as a multichannel bias field estimation technique itself.
Most of bias correction methodologies (e.g. \cite{tustison_n4itk:_2010}) have been well tested on
  \gls*{t1} images, but their behavior has not been studied in depth with other \gls*{mri} sequences.
In addition, they do not exploit the advantages of the underlying distribution model shared
  among pulse sequences.
The fine tuning of the bias estimation strategy included in \gls*{mbis}, and the demonstration
  of its expedience for the bias correction of multivariate images is a promising
  line for forthcoming research.

The robustness against bias field inhomogeneity exhibited by the multivariate segmentation technique
  was an illustrative example for promoting the use of multivariate approaches in neuroimaging.

\subsection{Repeatability analysis}
We tested the robustness of the presented tool against the variabilities that are intrinsic,
  mainly, to the acquisition.
We used the \emph{Kirby21 database}, consisting of 21 scan-rescan sessions on 21 healthy subjects
  with a multivariate protocol.
We attempted a number of combinations of the most suitable modalities
  (\gls*{t1}, \gls*{t2} and \gls*{mt}) for our experiment.
The results highlighted three important limitations in the experiments:
1) The quality of the image channels affects the sensitivity and robustness achieved in the 
   experiment.
2) Registration between observed variables ideally needs to be perfect.
  We briefly studied this drawback within the first experiment and present the
  impact in \autoref{fig:segmentation_pitfalls}.
  Despite this deteriorating event, the results remained satisfactory.
3) The measured increments of \acrfull*{icv} fractions strongly relied
   on the brain mask obtained with the brain extraction tool.
   Important differences in the volume of this mask
   biased the quantitative results. Less impact should be
   expected in the comparison between modality combinations.

The intention was to replicate the robustness analyses proposed elsewhere
  \citep{de_boer_accuracy_2010}.
However, one of our design considerations was to present results
  on publicly available data.
As we use a different database, the results presented in this work are not
  directly comparable to this previous study.
  
\subsection{Aptness of \gls*{mbis} in large-scale studies}
We performed a large-scale study with 585 cases, from a freely available database.
This experiment showed the expedience of \gls*{mbis} for the robust segmentation of large
  volumes of data, producing sound results.
Of note, our experiment was not intended to contribute to the field of study of brain aging,
  but we proved that \gls*{mbis} can be used for this purpose in a large-scale study with
  multivariate data (i.e. \citep{hodneland_automated_2012}).
Further work may prove that this multivariate approach is better than traditional \gls*{t1}-based 
  analysis, but in this case, we were restricted by the public availability of datasets.
As addressed in \autoref{sec:potentials_multivariate} below, this paper is intended to open
  the discussion of the potential benefits of multivariate analysis.
By selecting the appropriate MR contrasts, and developing new models for the selected multivariate
  distributions, the results of this last experiment should significantly outperform the classical
  monospectral approach.

\subsection{Potential of multivariate segmentation}
\label{sec:potentials_multivariate}
Besides the free availability of the presented tool and its evaluation,
  the most interesting result was the potential robustness suggested by
  multivariate segmentation.
There are still some challenging issues in brain tissue segmentation, for
  example the need for precise delineation of deep \gls*{gm} structures.
Some efforts have been devoted to deep brain nuclei segmentation 
  \citep{pohl_hierarchical_2007,tu_brain_2008}, but this
  application was beyond the scope of this work, given that, in general, brain
  tissue segmentation is not aimed at identifying the nuclei.
Some studies more aligned with \gls*{mbis} foundations proposed new acquisition
  sequences \citep{marques_mp2rage_2010,west_novel_2012} or the use of some 
  other existing ones \citep{helms_improved_2009} to overcome this issue.
Many of these sequences are acquired implicitly registered with other modalities,
  while some are inherently multichannel, which necessitates fully supportive
  multivariate segmentation.
Consequently, the results presented in this paper using well-established modalities could
  be improved by those obtained with the aforementioned emerging
  modalities and multivariate sequences.

The robustness issued by priors in atlas-based methods can be achieved
  with multivariate segmentation without atlases, overcoming the drawbacks of
  monospectral data-driven methods.
As mentioned in \autoref{sec:introduction}, atlas-free Bayesian segmentation methods
  can be directly applied in the clinical assessment of several global pathologies (e.g.
  atrophy, degeneration, enlarged ventricles) without modifications.
In focal conditions (e.g. tumors, multiple sclerosis, white matter lesions),
  the main requirement is the adaptation of the model used in normal subjects to the
  pathology, or including outlier rejection schemes \citep{van_leemput_automated_2001}.
In this context, \gls*{mbis} is certainly a potentially useful tool, given its
  availability and its flexibility for the necessary adaptations.


\section{Conclusion}
This work presented new and flexible segmentation software that is
  intended as the basis for a large statistical clustering
  suite for biomedical imaging.
To this end, a first version of the \gls*{mbis} tool has been made publicly 
  available, providing clinical researchers with a complete and functional
 software tool and a complementary testing framework which includes the
 presented experiments.
We also presented \gls*{mbis} to encourage multivariate analysis of data,
  as an emerging set of methodologies that should eventually
  improve the repeatability of segmentation procedures.
Complexity sources on multivariate statistical clustering are plentiful, with
  numerous alternatives, such as the mixture model selection,
  the \gls*{markov} model optimization, the bias field estimation 
  and correction, and the use of atlases.
The presented first release of \gls*{mbis} supports $n$-class segmentation using
  the \gls*{e_m} algorithm, with a novel bias modeling approach, and
  \gls*{markov} model regularization solved with \gls*{graph_cuts}
  optimization.
We evaluated the accuracy and robustness of \gls*{mbis} to demonstrate its usefulness.
Finally, we understand that \gls*{mbis} will be useful for both computer vision as
  well as clinical communities; we also hope that it will eventually encourage
  investigators to enhance further the capabilities of this publicly available 
  research tool.